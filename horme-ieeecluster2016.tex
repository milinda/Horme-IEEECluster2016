% Title     :: SamzaSQL: Scalable Fast Data Management with Streaming SQL
% Author    :: Milinda Pathirage
% Email     :: mpathira@indiana.edu
% Website   :: http://milinda.pathirage.org
% Template  :: sthlm beamer theme by Benjamin Weiss (hendryolson@gmail.com, http://v42.com), 
%              which is HEAVILY based on the HSRM beamer theme created by Benjamin Weiss
%			   (benjamin.weiss@student.hs-rm.de), which can be found on GitHub
%			   <https://github.com/hsrmbeamertheme/hsrmbeamertheme>.



%-=-=-=-=-=-=-=-=-=-=-=-=-=-=-=-=-=-=-=-=-=-=-=-=
%
%        LOADING DOCUMENT
%
%-=-=-=-=-=-=-=-=-=-=-=-=-=-=-=-=-=-=-=-=-=-=-=-=

\documentclass[newPxFont]{beamer}
\usetheme{sthlm}
%\usecolortheme{sthlmv42}

%-=-=-=-=-=-=-=-=-=-=-=-=-=-=-=-=-=-=-=-=-=-=-=-=
%        LOADING PACKAGES
%-=-=-=-=-=-=-=-=-=-=-=-=-=-=-=-=-=-=-=-=-=-=-=-=
\usepackage[utf8]{inputenc}
\usepackage{hyperref}
\usepackage{minted}
\usepackage{xcolor}
\usepackage{tikz}
\usepackage{tikz-qtree}
\usetikzlibrary{trees}
\usepackage{xxcolor}
\usetikzlibrary{shapes.misc,shapes.geometric,shapes.arrows,decorations.pathmorphing,decorations.shapes}
\usetikzlibrary{matrix,chains,scopes,positioning,arrows,fit}


\usepackage{chronology}

\renewcommand{\event}[3][e]{%
  \pgfmathsetlength\xstop{(#2-\theyearstart)*\unit}%
  \ifx #1e%
    \draw[fill=black,draw=none,opacity=0.5]%
      (\xstop, 0) circle (.2\unit)%
      node[opacity=1,rotate=45,right=.2\unit] {#3};%
  \else%
    \pgfmathsetlength\xstart{(#1-\theyearstart)*\unit}%
    \draw[fill=black,draw=none,opacity=0.5,rounded corners=.1\unit]%
      (\xstart,-.1\unit) rectangle%
      node[opacity=1,rotate=45,right=.2\unit] {#3} (\xstop,.1\unit);%
  \fi}%

%-=-=-=-=-=-=-=-=-=-=-=-=-=-=-=-=-=-=-=-=-=-=-=-=
%        BEAMER OPTIONS
%-=-=-=-=-=-=-=-=-=-=-=-=-=-=-=-=-=-=-=-=-=-=-=-=

%\setbeameroption{show notes}

%-=-=-=-=-=-=-=-=-=-=-=-=-=-=-=-=-=-=-=-=-=-=-=-=
%
%	PRESENTATION INFORMATION
%
%-=-=-=-=-=-=-=-=-=-=-=-=-=-=-=-=-=-=-=-=-=-=-=-=

\title{Horme}
\subtitle{Random Access Big Data Analytics}
%\date{\small{\jobname}}
%\date{\today}
\author{Guangchen Ruan, Beth Plale; \emph{Milinda Pathirage (Presenter)}}
\institute{School of Informatics and Computing, Indiana University}

\hypersetup{
pdfauthor = {Milinda Pathirage: milinda.pathirage@gmail.com},
pdfsubject = {},
pdfkeywords = {},
pdfmoddate= {D:\pdfdate},
pdfcreator = {}
}

\begin{document}
\setbeamertemplate{caption}{\raggedright\insertcaption\par}

%-=-=-=-=-=-=-=-=-=-=-=-=-=-=-=-=-=-=-=-=-=-=-=-=
%
%	TITLE PAGE
%
%-=-=-=-=-=-=-=-=-=-=-=-=-=-=-=-=-=-=-=-=-=-=-=-=

\maketitle

%\begin{frame}[plain]
%	\titlepage
%\end{frame}

%-=-=-=-=-=-=-=-=-=-=-=-=-=-=-=-=-=-=-=-=-=-=-=-=
%
%	TABLE OF CONTENTS: OVERVIEW
%
%-=-=-=-=-=-=-=-=-=-=-=-=-=-=-=-=-=-=-=-=-=-=-=-=

\section{Introduction}

%-=-=-=-=-=-=-=-=-=-=-=-=-=-=-=-=-=-=-=-=-=-=-=-=
%	FRAME:
%-=-=-=-=-=-=-=-=-=-=-=-=-=-=-=-=-=-=-=-=-=-=-=-=
\begin{frame}[c]{Hathitrust Research Center (HTRC)}
Research arm of \textbf{Hathitrust Digital Library} that develops systems for enabling computational access to 13.3 million digitized volumes.

\vspace{1em}

\begin{exampleblock}{Mission}
Enable researchers world-wide to accomplish tera-scale text data-mining and analysis  
\end{exampleblock}
\end{frame}

\begin{frame}[c]{HTRC Platform}

% Based on http://tex.stackexchange.com/questions/125234/upside-down-tikz-qtree-with-concentrated-edges
\tikzstyle{var} = [draw,shape=rectangle,minimum size=1em,rounded corners=1mm]
\tikzstyle{operator} = [draw=none,fill=none,above,pos=0]

\begin{tikzpicture}
[   grow'=up,
    level distance=.8cm,
    sibling distance=.6cm,
    edge from parent fork up,
    edge from parent/.style={draw,rounded corners=1mm}
]
\begin{scope}[local bounding box=scope1]
  \draw[rounded corners=5pt] (-6.7,4.6) rectangle (-3.7,4) node[pos=.5] {\tiny Prepackaged Algorithms};
  \draw[rounded corners=5pt] (-2.85,4.6) rectangle (0.15,4) node[pos=0.5] {\tiny Data Capsules};
  \draw[rounded corners=5pt] (1,4.6) rectangle (4,4) node[pos=0.5] {\tiny Big Data Analytics};
  \draw[fill={rgb:orange,1;yellow,2;pink,5},rounded corners=5pt] (-5.7,3.6) rectangle (4,3) ;
  \draw[fill={rgb:orange,1;yellow,2;pink,5},rounded corners=5pt] (-6.2,3.3) rectangle (3.5,2.7) ;
  \draw[fill={rgb:orange,1;yellow,2;pink,5},rounded corners=5pt] (-6.7,3) rectangle (3,2.4) node[pos=.5] {\tiny Corpus Abstractions (KV, REST, \textbf{Horme})};
\end{scope}

\begin{scope}[shift={($(scope1.north)+(0.5cm,-5cm)$)}]
  \tikzset{every node/.style={var}}
  \Tree [.pairtree-root 
            [\edge node[operator] {};
            [
              [.\node[] {}; ] [.\node[] {}; ]
             ] [ [.\node[] {}; ][.\node[] {}; ]] [ [.\node[] {}; ][.\node[dotted] {}; ][.\node[] {}; ] ]
          ]
          [ [[.\node[] {}; ] [.\node[dotted] {}; ] [.\node[dotted] {}; ] [.\node[] {}; ]] 
          ]
        ]
\end{scope}
\end{tikzpicture}
\end{frame}

\end{document}